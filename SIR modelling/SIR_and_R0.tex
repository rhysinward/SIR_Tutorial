% Options for packages loaded elsewhere
\PassOptionsToPackage{unicode}{hyperref}
\PassOptionsToPackage{hyphens}{url}
%
\documentclass[
]{article}
\usepackage{amsmath,amssymb}
\usepackage{iftex}
\ifPDFTeX
  \usepackage[T1]{fontenc}
  \usepackage[utf8]{inputenc}
  \usepackage{textcomp} % provide euro and other symbols
\else % if luatex or xetex
  \usepackage{unicode-math} % this also loads fontspec
  \defaultfontfeatures{Scale=MatchLowercase}
  \defaultfontfeatures[\rmfamily]{Ligatures=TeX,Scale=1}
\fi
\usepackage{lmodern}
\ifPDFTeX\else
  % xetex/luatex font selection
\fi
% Use upquote if available, for straight quotes in verbatim environments
\IfFileExists{upquote.sty}{\usepackage{upquote}}{}
\IfFileExists{microtype.sty}{% use microtype if available
  \usepackage[]{microtype}
  \UseMicrotypeSet[protrusion]{basicmath} % disable protrusion for tt fonts
}{}
\makeatletter
\@ifundefined{KOMAClassName}{% if non-KOMA class
  \IfFileExists{parskip.sty}{%
    \usepackage{parskip}
  }{% else
    \setlength{\parindent}{0pt}
    \setlength{\parskip}{6pt plus 2pt minus 1pt}}
}{% if KOMA class
  \KOMAoptions{parskip=half}}
\makeatother
\usepackage{xcolor}
\usepackage[margin=1in]{geometry}
\usepackage{color}
\usepackage{fancyvrb}
\newcommand{\VerbBar}{|}
\newcommand{\VERB}{\Verb[commandchars=\\\{\}]}
\DefineVerbatimEnvironment{Highlighting}{Verbatim}{commandchars=\\\{\}}
% Add ',fontsize=\small' for more characters per line
\usepackage{framed}
\definecolor{shadecolor}{RGB}{248,248,248}
\newenvironment{Shaded}{\begin{snugshade}}{\end{snugshade}}
\newcommand{\AlertTok}[1]{\textcolor[rgb]{0.94,0.16,0.16}{#1}}
\newcommand{\AnnotationTok}[1]{\textcolor[rgb]{0.56,0.35,0.01}{\textbf{\textit{#1}}}}
\newcommand{\AttributeTok}[1]{\textcolor[rgb]{0.13,0.29,0.53}{#1}}
\newcommand{\BaseNTok}[1]{\textcolor[rgb]{0.00,0.00,0.81}{#1}}
\newcommand{\BuiltInTok}[1]{#1}
\newcommand{\CharTok}[1]{\textcolor[rgb]{0.31,0.60,0.02}{#1}}
\newcommand{\CommentTok}[1]{\textcolor[rgb]{0.56,0.35,0.01}{\textit{#1}}}
\newcommand{\CommentVarTok}[1]{\textcolor[rgb]{0.56,0.35,0.01}{\textbf{\textit{#1}}}}
\newcommand{\ConstantTok}[1]{\textcolor[rgb]{0.56,0.35,0.01}{#1}}
\newcommand{\ControlFlowTok}[1]{\textcolor[rgb]{0.13,0.29,0.53}{\textbf{#1}}}
\newcommand{\DataTypeTok}[1]{\textcolor[rgb]{0.13,0.29,0.53}{#1}}
\newcommand{\DecValTok}[1]{\textcolor[rgb]{0.00,0.00,0.81}{#1}}
\newcommand{\DocumentationTok}[1]{\textcolor[rgb]{0.56,0.35,0.01}{\textbf{\textit{#1}}}}
\newcommand{\ErrorTok}[1]{\textcolor[rgb]{0.64,0.00,0.00}{\textbf{#1}}}
\newcommand{\ExtensionTok}[1]{#1}
\newcommand{\FloatTok}[1]{\textcolor[rgb]{0.00,0.00,0.81}{#1}}
\newcommand{\FunctionTok}[1]{\textcolor[rgb]{0.13,0.29,0.53}{\textbf{#1}}}
\newcommand{\ImportTok}[1]{#1}
\newcommand{\InformationTok}[1]{\textcolor[rgb]{0.56,0.35,0.01}{\textbf{\textit{#1}}}}
\newcommand{\KeywordTok}[1]{\textcolor[rgb]{0.13,0.29,0.53}{\textbf{#1}}}
\newcommand{\NormalTok}[1]{#1}
\newcommand{\OperatorTok}[1]{\textcolor[rgb]{0.81,0.36,0.00}{\textbf{#1}}}
\newcommand{\OtherTok}[1]{\textcolor[rgb]{0.56,0.35,0.01}{#1}}
\newcommand{\PreprocessorTok}[1]{\textcolor[rgb]{0.56,0.35,0.01}{\textit{#1}}}
\newcommand{\RegionMarkerTok}[1]{#1}
\newcommand{\SpecialCharTok}[1]{\textcolor[rgb]{0.81,0.36,0.00}{\textbf{#1}}}
\newcommand{\SpecialStringTok}[1]{\textcolor[rgb]{0.31,0.60,0.02}{#1}}
\newcommand{\StringTok}[1]{\textcolor[rgb]{0.31,0.60,0.02}{#1}}
\newcommand{\VariableTok}[1]{\textcolor[rgb]{0.00,0.00,0.00}{#1}}
\newcommand{\VerbatimStringTok}[1]{\textcolor[rgb]{0.31,0.60,0.02}{#1}}
\newcommand{\WarningTok}[1]{\textcolor[rgb]{0.56,0.35,0.01}{\textbf{\textit{#1}}}}
\usepackage{graphicx}
\makeatletter
\def\maxwidth{\ifdim\Gin@nat@width>\linewidth\linewidth\else\Gin@nat@width\fi}
\def\maxheight{\ifdim\Gin@nat@height>\textheight\textheight\else\Gin@nat@height\fi}
\makeatother
% Scale images if necessary, so that they will not overflow the page
% margins by default, and it is still possible to overwrite the defaults
% using explicit options in \includegraphics[width, height, ...]{}
\setkeys{Gin}{width=\maxwidth,height=\maxheight,keepaspectratio}
% Set default figure placement to htbp
\makeatletter
\def\fps@figure{htbp}
\makeatother
\setlength{\emergencystretch}{3em} % prevent overfull lines
\providecommand{\tightlist}{%
  \setlength{\itemsep}{0pt}\setlength{\parskip}{0pt}}
\setcounter{secnumdepth}{-\maxdimen} % remove section numbering
\ifLuaTeX
  \usepackage{selnolig}  % disable illegal ligatures
\fi
\IfFileExists{bookmark.sty}{\usepackage{bookmark}}{\usepackage{hyperref}}
\IfFileExists{xurl.sty}{\usepackage{xurl}}{} % add URL line breaks if available
\urlstyle{same}
\hypersetup{
  pdftitle={SIR Model and R0 Calculation},
  pdfauthor={Rhys Inward},
  hidelinks,
  pdfcreator={LaTeX via pandoc}}

\title{SIR Model and R0 Calculation}
\author{Rhys Inward}
\date{2024-11-25}

\begin{document}
\maketitle

\hypertarget{sir-model-and-r0-calculation-in-r}{%
\section{SIR Model and R0 Calculation in
R}\label{sir-model-and-r0-calculation-in-r}}

\hypertarget{introduction}{%
\subsection{Introduction}\label{introduction}}

In this tutorial, we will explore the SIR
(Susceptible-Infectious-Recovered) model, a fundamental mathematical
model in epidemiology used to understand the spread of infectious
diseases. The objectives of this tutorial are:

\begin{itemize}
\tightlist
\item
  Explain what the basic reproduction number (R0) is.
\item
  Introduce the SIR model and how it can be used to calculate R0.
\item
  Investigate how changes in the transmission rate (\(\beta\)) and
  recovery rate (\(\gamma\)) affect R0 and the epidemic dynamics.
\item
  Introduce an intervention and analyse how changing its timing impacts
  the overall epidemic size.
\end{itemize}

By the end of this tutorial, you will have a deeper understanding of
infectious disease modeling and the critical factors that influence
epidemic outcomes.

\begin{center}\rule{0.5\linewidth}{0.5pt}\end{center}

\hypertarget{what-is-r0}{%
\subsection{1. What is R0?}\label{what-is-r0}}

The \textbf{basic reproduction number}, denoted as \textbf{R0}, is a key
epidemiological metric that represents the average number of secondary
infections produced by one infected individual in a completely
susceptible population.

\begin{itemize}
\tightlist
\item
  \textbf{If R0 \textgreater{} 1}: Each infected person infects more
  than one person on average, leading to the potential for an epidemic.
\item
  \textbf{If R0 \textless{} 1}: The infection will likely die out over
  time.
\end{itemize}

Understanding R0 helps in assessing the potential for disease spread and
in planning control strategies.

\begin{center}\rule{0.5\linewidth}{0.5pt}\end{center}

\hypertarget{the-sir-model}{%
\subsection{2. The SIR Model}\label{the-sir-model}}

The \textbf{SIR model} is a simple mathematical model to simulate how a
disease spreads through a population. It divides the population into
three compartments:

\begin{itemize}
\tightlist
\item
  \textbf{S (Susceptible)}: Individuals who are susceptible to the
  disease.
\item
  \textbf{I (Infectious)}: Individuals who are infected and can spread
  the disease.
\item
  \textbf{R (Recovered)}: Individuals who have recovered from the
  disease and are immune (can no longer transmit).
\end{itemize}

The transitions between these compartments are governed by two
parameters:

\begin{itemize}
\tightlist
\item
  \textbf{\(\beta\) (beta)}: The transmission rate, representing the
  probability of transmission per contact - i.e.~how fast a disease
  spreads.
\item
  \textbf{\(\gamma\) (gamma)}: The recovery rate, representing the rate
  at which infected individuals recover and gain immunity.
\end{itemize}

The model is described by the following differential equations:

\[
\begin{align*}
\frac{dS}{dt} &= -\beta \frac{S I}{N} \\
\frac{dI}{dt} &= \beta \frac{S I}{N} - \gamma I \\
\frac{dR}{dt} &= \gamma I
\end{align*}
\]

Where \(N = S + I + R\) is the total population.

\begin{center}\rule{0.5\linewidth}{0.5pt}\end{center}

\hypertarget{calculating-r0-from-sir-model-parameters}{%
\subsection{3. Calculating R0 from SIR Model
Parameters}\label{calculating-r0-from-sir-model-parameters}}

In the context of the SIR model, the basic reproduction number R0 is
calculated as:

\[
R_0 = \frac{\beta}{\gamma}
\]

This formula shows that R0 increases with a higher transmission rate and
decreases with a higher recovery rate.

\begin{center}\rule{0.5\linewidth}{0.5pt}\end{center}

\hypertarget{implementing-the-sir-model}{%
\subsection{4. Implementing the SIR
Model}\label{implementing-the-sir-model}}

\hypertarget{setup}{%
\subsubsection{Setup}\label{setup}}

Ensure that you have the necessary packages installed and loaded.

\begin{Shaded}
\begin{Highlighting}[]
\CommentTok{\# Install packages if not already installed}
\NormalTok{required\_packages }\OtherTok{\textless{}{-}} \FunctionTok{c}\NormalTok{(}\StringTok{"deSolve"}\NormalTok{, }\StringTok{"ggplot2"}\NormalTok{, }\StringTok{"reshape2"}\NormalTok{)}

\NormalTok{installed\_packages }\OtherTok{\textless{}{-}} \FunctionTok{rownames}\NormalTok{(}\FunctionTok{installed.packages}\NormalTok{())}
\ControlFlowTok{for}\NormalTok{(p }\ControlFlowTok{in}\NormalTok{ required\_packages)\{}
  \ControlFlowTok{if}\NormalTok{(}\SpecialCharTok{!}\NormalTok{(p }\SpecialCharTok{\%in\%}\NormalTok{ installed\_packages))\{}
    \FunctionTok{install.packages}\NormalTok{(p, }\AttributeTok{dependencies =} \ConstantTok{TRUE}\NormalTok{)}
\NormalTok{  \}}
\NormalTok{\}}


\CommentTok{\# Load the packages}
\FunctionTok{library}\NormalTok{(deSolve)}
\FunctionTok{library}\NormalTok{(ggplot2)}
\FunctionTok{library}\NormalTok{(reshape2)}
\end{Highlighting}
\end{Shaded}

\hypertarget{define-model-parameters}{%
\subsubsection{Define Model Parameters}\label{define-model-parameters}}

\begin{Shaded}
\begin{Highlighting}[]
\CommentTok{\# Total population}
\NormalTok{N }\OtherTok{\textless{}{-}} \FloatTok{1e5}  \CommentTok{\# 100,000 individuals}

\CommentTok{\# Initial number of infected and recovered individuals}
\NormalTok{I0 }\OtherTok{\textless{}{-}} \DecValTok{1}
\NormalTok{R0\_initial }\OtherTok{\textless{}{-}} \DecValTok{0}

\CommentTok{\# Initial number of susceptible individuals}
\NormalTok{S0 }\OtherTok{\textless{}{-}}\NormalTok{ N }\SpecialCharTok{{-}}\NormalTok{ I0 }\SpecialCharTok{{-}}\NormalTok{ R0\_initial}

\CommentTok{\# Transmission rate (beta) and recovery rate (gamma)}
\NormalTok{beta }\OtherTok{\textless{}{-}} \FloatTok{0.3}    \CommentTok{\# Transmission rate}
\NormalTok{gamma }\OtherTok{\textless{}{-}} \FloatTok{0.1}   \CommentTok{\# Recovery rate}

\CommentTok{\# Calculate R0}
\NormalTok{R0\_value }\OtherTok{\textless{}{-}}\NormalTok{ beta }\SpecialCharTok{/}\NormalTok{ gamma}
\FunctionTok{cat}\NormalTok{(}\StringTok{"The basic reproduction number R0 is:"}\NormalTok{, R0\_value, }\StringTok{"}\SpecialCharTok{\textbackslash{}n}\StringTok{"}\NormalTok{)}
\end{Highlighting}
\end{Shaded}

\begin{verbatim}
## The basic reproduction number R0 is: 3
\end{verbatim}

\begin{Shaded}
\begin{Highlighting}[]
\CommentTok{\# Time points (in days)}
\NormalTok{times }\OtherTok{\textless{}{-}} \FunctionTok{seq}\NormalTok{(}\DecValTok{0}\NormalTok{, }\DecValTok{160}\NormalTok{, }\AttributeTok{by =} \DecValTok{1}\NormalTok{)}
\end{Highlighting}
\end{Shaded}

\begin{Shaded}
\begin{Highlighting}[]
\NormalTok{sir\_model }\OtherTok{\textless{}{-}} \ControlFlowTok{function}\NormalTok{(time, state, parameters) \{}
  \FunctionTok{with}\NormalTok{(}\FunctionTok{as.list}\NormalTok{(}\FunctionTok{c}\NormalTok{(state, parameters)), \{}
    \CommentTok{\# Rates of change}
\NormalTok{    dS }\OtherTok{\textless{}{-}} \SpecialCharTok{{-}}\NormalTok{beta }\SpecialCharTok{*}\NormalTok{ S }\SpecialCharTok{*}\NormalTok{ I }\SpecialCharTok{/}\NormalTok{ N}
\NormalTok{    dI }\OtherTok{\textless{}{-}}\NormalTok{ beta }\SpecialCharTok{*}\NormalTok{ S }\SpecialCharTok{*}\NormalTok{ I }\SpecialCharTok{/}\NormalTok{ N }\SpecialCharTok{{-}}\NormalTok{ gamma }\SpecialCharTok{*}\NormalTok{ I}
\NormalTok{    dR }\OtherTok{\textless{}{-}}\NormalTok{ gamma }\SpecialCharTok{*}\NormalTok{ I}

    \CommentTok{\# Return the rates of change}
    \FunctionTok{list}\NormalTok{(}\FunctionTok{c}\NormalTok{(dS, dI, dR))}
\NormalTok{  \})}
\NormalTok{\}}

\CommentTok{\# Initial state vector}
\NormalTok{init\_state }\OtherTok{\textless{}{-}} \FunctionTok{c}\NormalTok{(}\AttributeTok{S =}\NormalTok{ S0, }\AttributeTok{I =}\NormalTok{ I0, }\AttributeTok{R =}\NormalTok{ R0\_initial)}

\CommentTok{\# Parameters vector}
\NormalTok{parameters }\OtherTok{\textless{}{-}} \FunctionTok{c}\NormalTok{(}\AttributeTok{beta =}\NormalTok{ beta, }\AttributeTok{gamma =}\NormalTok{ gamma)}


\CommentTok{\# Solve the system using ode solver}
\NormalTok{sir\_output }\OtherTok{\textless{}{-}} \FunctionTok{ode}\NormalTok{(}\AttributeTok{y =}\NormalTok{ init\_state, }\AttributeTok{times =}\NormalTok{ times, }\AttributeTok{func =}\NormalTok{ sir\_model, }\AttributeTok{parms =}\NormalTok{ parameters)}

\CommentTok{\# Convert output to data frame}
\NormalTok{sir\_output }\OtherTok{\textless{}{-}} \FunctionTok{as.data.frame}\NormalTok{(sir\_output)}
\end{Highlighting}
\end{Shaded}

\hypertarget{plotting-the-results}{%
\subsubsection{Plotting the results}\label{plotting-the-results}}

\begin{Shaded}
\begin{Highlighting}[]
\CommentTok{\# Reshape data for plotting}
\NormalTok{sir\_long }\OtherTok{\textless{}{-}} \FunctionTok{melt}\NormalTok{(sir\_output, }\AttributeTok{id =} \StringTok{"time"}\NormalTok{, }\AttributeTok{measure =} \FunctionTok{c}\NormalTok{(}\StringTok{"S"}\NormalTok{, }\StringTok{"I"}\NormalTok{, }\StringTok{"R"}\NormalTok{))}

\CommentTok{\# Plot the SIR curves}
\FunctionTok{ggplot}\NormalTok{(}\AttributeTok{data =}\NormalTok{ sir\_long, }\FunctionTok{aes}\NormalTok{(}\AttributeTok{x =}\NormalTok{ time, }\AttributeTok{y =}\NormalTok{ value, }\AttributeTok{color =}\NormalTok{ variable)) }\SpecialCharTok{+}
  \FunctionTok{geom\_line}\NormalTok{(}\AttributeTok{size =} \DecValTok{1}\NormalTok{) }\SpecialCharTok{+}
  \FunctionTok{labs}\NormalTok{(}
    \AttributeTok{x =} \StringTok{"Time (days)"}\NormalTok{,}
    \AttributeTok{y =} \StringTok{"Number of Individuals"}\NormalTok{,}
    \AttributeTok{title =} \StringTok{"SIR Model Simulation"}\NormalTok{,}
    \AttributeTok{color =} \StringTok{"Compartment"}
\NormalTok{  ) }\SpecialCharTok{+}
  \FunctionTok{theme\_minimal}\NormalTok{()}
\end{Highlighting}
\end{Shaded}

\includegraphics{SIR_and_R0_files/figure-latex/plot-1.pdf}

\end{document}
